#encoding utf-8

\documentclass[11pt,a4paper,sans]{moderncv}        % 'sans' and 'roman' for font family
\moderncvstyle{casual}                             % style options are 'casual' (default), 'classic', 'oldstyle' and 'banking'
\moderncvcolor{blue}                               % color options 'blue' (default), 'orange', 'green', 'red', 'purple', 'grey' and 'black'
%\nopagenumbers{}                                  % uncomment to suppress automatic page numbering for CVs longer than one page
\usepackage[utf8]{inputenc}                        % if you are not using xelatex ou lualatex, replace by the encoding you are using
\usepackage[scale=0.75,a4paper]{geometry}
\usepackage[english]{babel}
\usepackage{csquotes}

\setlength{\hintscolumnwidth}{3cm}
%----------------------------------------------------------------------------------
%            personal data
%----------------------------------------------------------------------------------
\firstname{$header.firstname}
\familyname{$header.familyname}
\address{${header.address.line1}${header.address.line2}}{$header.address.line3}{$header.address.line4}
\mobile{$header.phone}
\email{$header.email}
\homepage{$header.website}
\begin{document}
%-----       resume       ---------------------------------------------------------
\makecvtitle

\section{Education}
#for $entry in $education.iterchildren(tag='entry')
\cventry{$entry.from -- $entry.to}
  {$entry.organisation.name}
	{$entry.organisation.address}
	{}
	{}
	{\textbf{$entry.title}\\ $entry.detail}
#end for

\section{Professional Experience}
#for $entry in $professionalExperience.iterchildren(tag='entry')
\cventry{$entry.from -- $entry.to}
  {$entry.organisation.name}
	{\newline $entry.organisation.address}
	{}
	{}
	{\textbf{$entry.title}\\ $entry.detail}
#end for

\section{Language skills}
#for $language in $languages.iterchildren(tag='language')
\cvitemwithcomment{$language.name}{$language.level}{
#if $hasattr(language, 'comment') then $language.comment else ''
}
#end for

\section{Open Source Projects}
#for $project in $projects.iterchildren(tag='project')
\cvitem{$project.name}{$project.description}
#end for

\section{Computer Skills}
#for $skillGroup in $skillGroups.iterchildren(tag='skillGroup')
\cvitem{$skillGroup.name}{
#for $skill in $skillGroup.skills.iterchildren(tag='skill')
$skill.title $skill.detail
\newline
#end for
}
#end for

\section{Other}
\cvitem{Version of resumé}{\today\newline
	Most recent version: \url{$linkToResume}}
\end{document}
